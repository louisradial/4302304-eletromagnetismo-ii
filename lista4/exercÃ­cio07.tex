% vim: spl=pt
\begin{exercício}{Instabilidade do átomo de Bohr}{ex7}
    Um dos principais problemas do modelo atômico de Bohr é a noção de que cargas aceleradas emitem radiação eletromagnética, impossibilitando a existência de órbitas circulares estáveis. Vamos revisitar esse modelo considerando o átomo de hidrogênio em seu estado fundamental. Assumindo que o elétron se encontra inicialmente em um movimento circular de raio \(r = a_0\) em torno do próton, onde \(a_0\) é o raio de Bohr, estime o tempo que levaria para o elétron colidir com o núcleo.
\end{exercício}
\begin{proof}[Resolução]
    Obs: Estimativa, fórmula de Larmor não relativística, teorema do Virial.
\end{proof}
