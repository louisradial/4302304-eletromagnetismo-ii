% vim: spl=pt
\begin{exercício}{Distribuição de fótons emitidos por radiação Cherenkov}{ex8}
    No contexto da radiação Cherenkov, a perda de energia por unidade de comprimento percorrido é dada por
    \begin{equation*}
        \diff{E_\mathrm{rad}}{x_q} = \frac{q^2}{4\pi \epsilon_0 c^2} \int_{\beta^2 \epsilon(\omega) > 1} \dli{\omega}\omega\left(1 - \frac{1}{\beta^2 \epsilon(\omega)}\right),
    \end{equation*}
    expressão conhecida como fórmula de Frank-Tamm. Quanticamente, a energia perdida por radiação eletromagnética em uma faixa de frequência entre \(\omega\) e \(\omega + \dl{\omega}\) é dada por \(\hbar \omega \dl{N},\) em que \(\dl{N}\) é o número de fótons emitidos. Deduza, a partir de uma análise da expressão de Frank-Tamm, uma expressão para o número total de fótons emitidos com comprimentos de onda entre \(\lambda_1\) e \(\lambda_2,\) com \(\lambda_2 > \lambda_1,\) por unidade de comprimento percorrido.
\end{exercício}
\begin{proof}[Resolução]
    Escrevendo \(q = Ze,\) e usando a constante de estrutura fina \(\alpha = \frac{e^2}{4\pi \epsilon_0 \hbar c},\) a fórmula de Frank-Tamm pode ser reescrita como
    \begin{equation*}
        \diff{E_\mathrm{rad}}{x_q} = \frac{\hbar \alpha Z^2}{c} \int_{\beta^2 \epsilon(\omega) > 1} \dli{\omega}\omega\left(1 - \frac{1}{\beta^2 \epsilon(\omega)}\right).
    \end{equation*}
    A energia perdida por radiação por unidade de comprimento por unidade de frequência é dada por
    \begin{equation*}
        \diff{E_{\mathrm{rad}}}{x_q,\omega} = \hbar \omega \diff{N}{x_q,\omega},
    \end{equation*}
    onde \(N\) é o número de fótons emitidos. Assim, o número de fótons emitidos por unidade de comprimento é dado por
    \begin{align*}
        \diff{N}{x_q}[\omega]\dl{\omega} = \frac{\alpha Z^2}{c} \left(1 - \frac{1}{\beta^2 \epsilon(\omega)}\right) \dli{\omega} 
        &\implies \diff{N}{x_q} = \frac{\alpha Z^2}{c} \int_{\beta^2 \epsilon(\omega) > 1} \dli{\omega}\left(1 - \frac{1}{\beta^2 \epsilon(\omega)}\right)\\
        &\implies\diff{N}{x_q} = \frac{\alpha Z^2}{c} \int_{0}^{\infty} \dli{\omega}\left(1 - \frac{1}{\beta^2 \epsilon(\omega)}\right)\theta(\beta^2 \epsilon(\omega) - 1).
    \end{align*}
    Mudando de variáveis para o comprimento de onda e escrevendo \(\epsilon(\lambda) = \epsilon(\omega)\) para \(\lambda \omega = 2\pi c,\) temos
    \begin{equation*}
        \diff{N}{x_q} = 2\pi \alpha Z^2 \int_0^\infty \dli{\lambda} \frac{1 - \frac{1}{\beta^2 \epsilon(\lambda)}}{\lambda^2} \theta(\beta^2 \epsilon(\lambda) - 1),
    \end{equation*}
    portanto 
    \begin{equation*}
        \diff{N}{x_q}([\lambda_1, \lambda_2]) = 2\pi \alpha Z^2 \int_{\lambda_1}^{\lambda_2} \dli{\lambda} \frac{1 - \frac{1}{\beta^2 \epsilon(\lambda)}}{\lambda^2}\theta(\beta^2 \epsilon(\lambda) - 1)
    \end{equation*}
    é o número de fótons emitidos com comprimento de onda no intervalo \([\lambda_1, \lambda_2]\) por unidade de comprimento percorrido.
\end{proof}
