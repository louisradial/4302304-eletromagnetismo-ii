% vim: spl=pt
\begin{exercício}{Radiação de uma carga pontual em movimento harmônico simples}{ex4}
    Uma carga pontual \(q\) se move com movimento harmônico simples ao longo do eixo \(z,\) segundo \(z(t) = a \cos(\omega_0 t).\) Mostre que a potência instantânea irradiada por unidade de ângulo sólido é
    \begin{equation*}
        \diff{P_\mathrm{rad}(\tr)}{\Omega} = \frac{\mu_0 q^2 c \beta^4}{16 \pi^2 a^2} \frac{\sin^2 \theta \cos^2(\omega_0 \tr)}{\left[1 + \beta \cos\theta \sin(\omega_0 \tr)\right]^5},
    \end{equation*}
    onde \(\tr\) denota o instante retardado e \(\beta = \frac{a \omega_0}{c}.\) Em seguida, mostre que 
    \begin{equation*}
        \mean*{\diff{P_\mathrm{rad}(\tr)}{\Omega}} = \frac{\mu_0 q^2 c \beta^4}{128 \pi^2 a^2} \left[\frac{4 + \beta^2 \cos^2\theta}{(1 - \beta^2 \cos^2 \theta)^{\frac72}}\right] \sin^2\theta
    \end{equation*}
    é a média da potência irradiada por ângulo sólido sobre um período de oscilação.
\end{exercício}
\begin{proof}[Resolução]
    
\end{proof}
