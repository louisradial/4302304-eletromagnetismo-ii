% vim: spl=pt
\begin{exercício}{Energia emitida por radiação para um elétron em desaceleração constante}{ex5}
    Suponha um elétron desacelerando a uma taxa constante \(a\) partir de uma dada velocidade inicial \(v_0 \ll c\) até parar. Que fração da sua energia cinética inicial é emitida em radiação? O restante é absorvido pelo mecanismo responsável pela aceleração constante.
\end{exercício}
\begin{proof}[Resolução]
    Para uma desaceleração constante com velocidades não relativísticas, temos 
    \begin{equation*}
        E_{\mathrm{rad}} = \int_0^{\frac{v_0}{a}} \dli{t} P_{\mathrm{rad}} = \frac{\mu_0 q^2 a v_0}{6\pi c}
    \end{equation*}
    como a energia emitida por radiação no intervalo em que a partícula é desacelerada. Assim,
    \begin{equation*}
        \frac{E_\mathrm{rad}}{T} = \frac{\mu_0 q^2 a}{3\pi m v_0 c}
    \end{equation*}
    é a fração da energia inicial que é emitida em radiação.
\end{proof}
