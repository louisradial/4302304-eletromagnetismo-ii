% vim: spl=pt
\begin{exercício}{Potencia irradiada por um circuito circular}{ex3}
    Uma corrente \(I(t)\) flui ao longo de um anel circular. Deduza a expressão geral para a potencia irradiada, expressando sua resposta em termos do momento de dipolo magnético \(\vetor{m}(t)\) do circuito.
\end{exercício}
\begin{proof}[Resolução]
    Do \href{https://github.com/louisradial/4302303-eletromagnetismo-i/releases/tag/lista8}{Exercício 1 da Lista 8 de Eletromagnetismo I}, o dipolo magnético é dado por \(\vetor{m}(t) = I(t) \vetor{a}\), onde \(\vetor{a} = -\frac12 \oint_{\Gamma} \dl{\vetor{\ell}}\times \vetor{\x}\) é o vetor área do anel circular \(\Gamma.\)
\end{proof}
