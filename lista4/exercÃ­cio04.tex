% vim: spl=pt
\begin{exercício}{Radiação de uma carga pontual em movimento harmônico simples}{ex4}
    Uma carga pontual \(q\) se move com movimento harmônico simples ao longo do eixo \(z,\) segundo \(z(t) = A \cos(\omega_0 t).\) Mostre que a potência instantânea irradiada por unidade de ângulo sólido é
    \begin{equation*}
        \diff{P_\mathrm{rad}(\tr)}{\Omega} = \frac{\mu_0 q^2 c^3 \beta_0^4}{16 \pi^2 A^2} \frac{\sin^2 \theta \cos^2(\omega_0 \tr)}{\left[1 + \beta_0 \cos\theta \sin(\omega_0 \tr)\right]^5},
    \end{equation*}
    onde \(\tr\) denota o instante retardado e \(\beta_0 = \frac{A \omega_0}{c}.\) Em seguida, mostre que 
    \begin{equation*}
        \mean*{\diff{P_\mathrm{rad}(\tr)}{\Omega}} = \frac{\mu_0 q^2 c^3 \beta_0^4}{128 \pi^2 A^2} \left[\frac{4 + \beta_0^2 \cos^2\theta}{(1 - \beta_0^2 \cos^2 \theta)^{\frac72}}\right] \sin^2\theta
    \end{equation*}
    é a média da potência irradiada por ângulo sólido sobre um período de oscilação.
\end{exercício}
\begin{proof}[Resolução]
    Como \(\vetor{\x}_q(t) = A \cos(\omega_0 t) \vetor{e}_z,\) temos \(\vetor{\beta}(t) = -\beta_0 \sin(\omega_0 t) \vetor{e}_z\) e \(\vetor{a}(t) = - \omega_0^2 \vetor{\x}_q(t)\). Assim, como a aceleração e a velocidade são colineares, temos
    \begin{align*}
        \diff{P_{\mathrm{rad}}(\tr)}{\Omega} &= \frac{\mu_0 q^2 c }{16 \pi^2}\left[\frac{\norm{\vetor{a}}^2\sin^2\theta}{(1 - \beta\cos\theta)^5}\right]_\ret\\
                                             &= \frac{\mu_0 q^2 \omega_0^4 A^2 c}{16 \pi^2}\frac{\cos^2(\omega_0\tr) \sin^2\theta}{\left[1 + \beta_0 \sin(\omega_0\tr) \cos\theta\right]^5}\\
                                             &= \frac{\mu_0 q^2 c^3 \beta_0^4}{16 \pi^2 A^2} \frac{\sin^2\theta \cos^2(\omega_0 \tr)}{\left[1 + \beta_0 \cos\theta \sin(\omega_0 \tr)\right]^5}
    \end{align*}
    como a expressão da potência irradiada por unidade de ângulo sólido. Com isso, a média da potência irradiada por ângulo sólido sobre um período de oscilação é dada por
    \begin{equation*}
        \mean*{\diff{P_\mathrm{rad}(\tr)}{\Omega}} = \frac{\omega_0}{2\pi} \int_0^{2\pi \omega_0^{-1}} \dli{\tr} \diff{P_\mathrm{rad}(\tr)}{\Omega} = \frac{\mu_0 q^2 c^3 \beta_0^4\sin^2\theta}{32 \pi^3 A^2} I(\beta_0 \cos\theta),
    \end{equation*}
    onde 
    \begin{equation*}
        I(\alpha) = \int_{0}^{2\pi} \dli{\psi} \frac{\cos^2\psi}{(1 + \alpha \sin \psi)^5},
    \end{equation*}
    com \(\alpha \in (-1,1).\) 

    Notemos que para \(\alpha = 0,\) temos \(I(0) = \pi,\) e para \(\alpha \neq 0,\) podemos escrever
    \begin{equation*}
        I(\alpha) = -\frac{1}{4\alpha} \left.\frac{\cos\psi}{(1 + \alpha \sin\psi)^4}\right\rvert_0^{2\pi} - \frac{1}{4\alpha} \int_0^{2\pi} \dli{\psi} \frac{\sin\psi}{(1 + \alpha \sin \psi)^4} = \frac1{6 \alpha} \diff*{\left[\frac12\int_0^{2\pi}\dli\psi \frac{1}{(1 + \alpha \sin\psi)^3}\right]}{\alpha}.
    \end{equation*}
    Seja \(J(\alpha)\) dada pela expressão entre os colchetes, então com a substituição de Weierstrass \(\xi = \tan\frac{\psi}{2},\) temos \(\dl{\psi} = \frac{2}{1 + \xi^2} \dl{\xi}\) e \(\sin\psi = \frac{2 \xi}{1 + \xi^2},\) portanto
    \begin{equation*}
        J(\alpha) = \frac12 \int_{-\pi}^\pi \dli\xi \frac{1}{(1 + \alpha \sin\psi)^3} = \int_{\mathbb{R}} \dli{\xi} \frac{\frac{1}{1 + \xi^2}}{\left(1 + \frac{2 \alpha \xi}{1 + \xi^2}\right)^3} = \int_{\mathbb{R}}\dli{\xi}\frac{(1 + \xi^2)^2}{(1 + 2 \alpha \xi + \xi^2)^3}
    \end{equation*}
    onde inicialmente utilizamos a \(2\pi\)-periodicidade do integrando. Como \(\alpha \in (-1,1),\) os polos do integrando são \(\zeta = -\alpha + i \sqrt{1 - \alpha^2}\) e \(\bar{\zeta},\) com \(\Im(\xi_0) > 0,\) e podemos escrever
    \begin{equation*}
        J(\alpha) = \int_{\mathbb{R}} \dli\xi\frac{(1 + \xi^2)^2}{(\xi - \zeta)^3 (\xi - \bar{\zeta})^3}.
    \end{equation*}

    Consideramos agora a função \(f : \mathbb{C} \to \mathbb{C}\), onde \(f(\xi)\) é igual ao integrando na última expressão obtida para \(J(\alpha)\), e o caminho \(\Gamma\) dado pela união do segmento de reta no eixo real de \(\xi = -R\) até \(\xi = R > 1\) com o semi-círculo \(\Gamma_R\) centrado na origem, de raio \(R\), no semiplano \(\Im(\xi) > 0,\) então 
    \begin{equation*}
        \int_{\Gamma}\dli{\xi} f(\xi) = 2 \pi i \operatorname{Res}_f(\zeta) \implies J(\alpha) = 2\pi i \operatorname{Res}_{f}(\zeta) - \lim_{R \to \infty}\int_{\Gamma_R}\dli{\xi} f(\xi),
    \end{equation*}
    pelo teorema do resíduo de Cauchy. Notemos que a integral no arco se anula no limite \(R \to \infty,\) pois temos \(\abs{1 + \xi^2}\leq 1 + R^2\) e \(\abs{\xi^2 + 2 \alpha \xi + 1} \geq R^2 - 2R\abs{\alpha} - 1 \sim R^2\) para todo \(\xi \in \Gamma_R,\) logo 
    \begin{equation*}
        \abs*{\lim_{R \to\infty}{\int_{\Gamma_R}\dli{\xi} f(\xi)}} \leq \lim_{R \to \infty}{\int_{0}^{\pi} R\dli{\varphi} \abs{f(R e^{i\varphi})}} \leq \pi \lim_{R \to \infty}{\frac{R(1 + R^2)^2}{(R^2 - 2 \abs{\alpha}R - 1)^3}} = 0.
    \end{equation*}
    O resíduo de \(f\) no polo de ordem três \(\zeta\) é dado por
    \begin{align*}
        \operatorname{Res}_f(\zeta) &= \frac1{2!}\lim_{\xi \to \zeta}{\diff*[2]{\left[(\xi - \zeta)^3 f(\xi)\right]}{\xi} }\\
                                    &= \frac12 \lim_{\xi \to \zeta}{\diff*{\frac{4\xi(1 + \xi^2) (\xi - \bar{\zeta}) - 3(1 + \xi^2)^2}{(\xi - \bar{\zeta})^4}}{\xi}}\\
                                    % &= \frac12 \lim_{\xi \to \zeta}{} \diff*{}{\xi}\frac{(\xi^2 - 4\bar{\zeta} \xi - 3) (1 + \xi^2)}{(\xi - \bar{\zeta})^4}\\
                                      &= \frac12 \lim_{\xi \to \zeta}{\diff*{\frac{\xi^4 - 4 \bar{\zeta} \xi^3 - 2\xi^2 - 4 \bar{\zeta} \xi - 3}{(\xi - \bar{\zeta})^4}}{\xi}}\\
                                      % &= \frac12 \lim_{\xi\to\zeta}{\frac{(4\xi^3 - 12 \bar{\zeta} \xi^2 - 4\xi - 4 \bar{\zeta})(\xi - \bar{\zeta}) - 4(\xi^4 - 4 \bar{\zeta} \xi^3 - 2\xi^2 - 4 \bar{\zeta} \xi - 3)}{(\xi - \bar{\zeta})^5}}\\
                                      &= 2\lim_{\xi\to\zeta}{\frac{(\xi^3 - 3 \bar{\zeta} \xi^2 - \xi -  \bar{\zeta})(\xi - \bar{\zeta}) - (\xi^4 - 4 \bar{\zeta} \xi^3 - 2\xi^2 - 4 \bar{\zeta} \xi - 3)}{(\xi - \bar{\zeta})^5}}\\
                                      &= 2\lim_{\xi \to \zeta}{} \frac{3 + \xi^2 + \bar{\zeta}^2 + 4 \bar{\zeta} \xi + 3 \bar{\zeta}^2 \xi^2 }{(\xi - \bar{\zeta})^5}\\
                                      &= 2 \frac{3 + \zeta^2 + \bar{\zeta}^2 + 4 \abs{\zeta}^2 + 3 \abs{\zeta}^4}{(\zeta - \bar{\zeta})^5}\\
                                      &= \frac{10 + (\zeta + \bar{\zeta})^2 - 2 \abs{\zeta}^2}{16 i \Im(\zeta)^5}\\
                                      &= \frac{2 + \alpha^2}{4 i (1 - \alpha^2)^{\frac52}}.
    \end{align*}
    Com isso, obtemos
    \begin{equation*}
        J(\alpha) = \frac{\pi}{2} \left[\frac{2 + \alpha^2}{(1 - \alpha^2)^{\frac52}}\right]
    \end{equation*}
    e então
    \begin{equation*}
        I(\alpha) = \frac{\pi}{12 \alpha} \diff*{\frac{2 + \alpha^2}{(1 - \alpha^2)^{\frac52}}}{\alpha} = \frac{\pi}{12 \alpha} \left[\frac{2 \alpha (1 - \alpha^2) + 5\alpha(2 + \alpha^2)}{(1 - \alpha^2)^{\frac72}}\right] = \frac{\pi}{4} \frac{4 + \alpha^2}{(1 - \alpha^2)^{\frac72}},
    \end{equation*}
    que reproduz o resultado de \(I(0) = \pi.\) Assim, substituindo \(\alpha = \beta_0 \cos\theta,\) temos
    \begin{equation*}
        \mean*{\diff{P_\mathrm{rad}(\tr)}{\Omega}} = \frac{\mu_0 q^2 c^3 \beta_0^4}{128 \pi^2 A^2} \left[\frac{4 + \beta_0^2 \cos^2\theta}{(1 - \beta_0^2 \cos^2 \theta)^{\frac72}}\right] \sin^2\theta
    \end{equation*}
    como a expressão para a média da potência irradiada por ângulo sólido sobre um período de oscilação.
\end{proof}
