% vim: spl=pt
\begin{exercício}{Radiação de um anel circular isolante carregado girando}{ex1}
    Um anel circular de raio \(b\) feito de material isolante encontra-se no plano \(xy\), centrado na origem. O anel possui densidade linear de carga \(\lambda(\varphi) = \lambda_0 \sin \varphi,\) onde \(\lambda_0\) é uma constante real e \(\varphi\) é o ângulo azimutal. O anel é posto a girar com velocidade angular constante \(\omega\) em torno do eixo \(z.\) Calcule a potência irradiada.
\end{exercício}
\begin{proof}[Resolução]
    
\end{proof}
