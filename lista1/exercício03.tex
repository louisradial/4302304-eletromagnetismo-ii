% vim: spl=pt
\begin{exercício}{Guia de ondas de seção circular}{ex3}
    Considere um tubo cilíndrico de seção transversal \emph{circular} de raio \(R\), cujas paredes são feitas de um material de condutividade \(\sigma \gg \epsilon \omega\), de forma que possa ser \emph{aproximado} por um condutor perfeito. Nesse caso, obtenha os modos TM de propagação nesse tubo e a frequência de corte para cada modo, utilizando coordenadas cilíndricas.

    Deseja-se estimar as perdas Ôhmicas no tubo, melhorando, mesmo que em uma primeira aproximação, a hipótese de condutor perfeito para as paredes. Calcule a potência perdida por efeito Joule nas paredes por unidade de comprimento \(\diff{P_\mathrm{loss}}{z}\) considerando apenas o modo TM de menor frequência de corte para o cálculo.
\end{exercício}
\begin{proof}[Resolução]
    
\end{proof}
