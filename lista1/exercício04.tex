% vim: spl=pt
\begin{exercício}{Modo TEM em uma linha de transmissão coaxial}{ex4}
    Considere um sistema que consiste em uma casca cilíndrica infinitamente longa, de raio interno \(b\), e em um cilindro maciço infinitamente longo de raio \(a < b\), arranjados de forma coaxial.
    \begin{enumerate}[label=(\alph*)]
        \item Assuma que o material que constitui os objetos podem ser aproximados por condutores perfeitos e mostre que, na região \(a < s < b,\) pode existir um modo de propagação TEM e obtenha os campos transversais nesse caso. Deixe a resposta em termos da amplitude do campo magnético azimutal em \(s = a^+\).
        \item Calcule a potência média transmitida no tubo.
        \item Assumindo agora que a condutividade \(\sigma\) é muito grande, porém finita, obtenha o coeficiente de atenuação de potência devido às perdas Ôhmicas.
    \end{enumerate}
\end{exercício}
\begin{proof}[Resolução]
    
\end{proof}
