% vim: spl=pt
\begin{exercício}{Guia de ondas de seção retangular}{ex1}
    Considere um tubo de seção retangular cujos lados são \(a\) e \(b\), com \(a > b\), feito de um material condutor (aproximadamente perfeito). Considere que o interior é preenchido com um material dielétrico linear e isotrópico caracterizado pelas constantes \(\epsilon\) e \(\mu\). Obtenha os modos de propagação TM dentro do tubo e as respectivas frequências de corte.
\end{exercício}
\begin{proof}[Resolução]
    Para o modo TM, temos \((\nabla_t^2 + \gamma^2)E_z = 0\) e \(\left.E_z\right|_S = 0\), com \(\gamma^2 = \mu \epsilon \omega^2 - k^2\). Separando variáveis, \(E_z(x,y) = X(x) Y(y),\) com \(X(a) = X(0) = Y(0) = Y(b) = 0,\) temos
    \begin{equation*}
        \frac{X''}{X} + \frac{Y''}{Y} + \gamma^2 = 0.
    \end{equation*}
    Assim, existem constantes \(\gamma_x, \gamma_y \in \mathbb{R} \setminus \set{0}\) com \(\gamma_x^2 + \gamma_y^2 = \gamma^2\) tais que \(X'' + \gamma_x^2 X = 0\) e \(Y'' + \gamma_y^2 Y = 0\). Logo, das condições de contorno, obtemos
    \begin{equation*}
        X(x) = X_0 \sin\left(\frac{n_x\pi}{a}x\right)
        \quad\text{e}\quad
        Y(y) = Y_0 \sin\left(\frac{n_y\pi}{b}y\right),
    \end{equation*}
    onde \(n_x, n_y \in \mathbb{N}\), com \(\left(\frac{n_x\pi}{a}\right)^2 + \left(\frac{n_y \pi}{b}\right)^2 = \gamma^2\). Assim, a frequência de corte do modo \((n_x, n_y)\) é \(\omega_{c_{n_x, n_y}} = \pi v \sqrt{\left(\frac{n_x}{a}\right)^2 + \left(\frac{n_y}{b}\right)^2}\), onde \(v = \frac{1}{\sqrt{\mu \epsilon}}.\) Para o modo \((n_x, n_y)\), o campo longitudinal é dado por
    \begin{equation*}
        E_z(x,y) = E_{n_x, n_y} \sin\left(\frac{n_x \pi x}{a}\right)\sin\left(\frac{n_y \pi y}{b}\right)
    \end{equation*}
    e os campos transversais por
    \begin{equation*}
        \textstyle
        \vetor{E}_t = \frac{ik}{\gamma_{n_x, n_y}^2} \nabla_t E_z = \frac{i \pi k E_{n_x, n_y}}{\gamma^2_{n_x, n_y}}\left[\frac{n_x}{a}\cos\left(\frac{n_x \pi x}{a}\right)\sin\left(\frac{n_y \pi y}{b}\right)\vetor{e}_x + \frac{n_y}{b}\sin\left(\frac{n_x \pi x}{a}\right)\cos\left(\frac{n_y \pi y}{b}\right)\vetor{e}_y\right]
    \end{equation*}
    e
    \begin{equation*}
        \textstyle
        \vetor{B}_t = \frac{i \mu \epsilon \omega}{\gamma^2_{n_x, n_y}} \vetor{e}_z \times \nabla_t E_z = \frac{i \pi \mu \epsilon \omega E_{n_x, n_y}}{\gamma^2_{n_x, n_y}}\left[\frac{n_x}{a}\cos\left(\frac{n_x \pi x}{a}\right)\sin\left(\frac{n_y \pi y}{b}\right)\vetor{e}_y - \frac{n_y}{b}\sin\left(\frac{n_x \pi x}{a}\right)\cos\left(\frac{n_y \pi y}{b}\right)\vetor{e}_x\right],
    \end{equation*}
    portanto
    \begin{equation*}
        \textstyle
        \vetor{E}^\mathrm{TM}_{n_x, n_y} = E_{n_x, n_y} \sin \left(\frac{n_x \pi x}{a}\right)\sin\left(\frac{n_y \pi y}{b}\right)\left\{\frac{i \pi k E_{n_x, n_y}}{\gamma^2_{n_x, n_y}}\left[\frac{n_x}{a}\csc\left(\frac{n_x \pi x}{a}\right)\vetor{e}_x + \frac{n_y}{b}\csc\left(\frac{n_y \pi y}{b}\right)\vetor{e}_y\right] + \vetor{e}_z\right\}e^{i(kz - \omega t)}
    \end{equation*}
    e
    \begin{equation*}
        \textstyle
        \vetor{B}^\mathrm{TM}_{n_x, n_y} = \frac{i \pi \mu \epsilon \omega E_{n_x, n_y}}{\gamma^2_{n_x, n_y}}\left[\frac{n_x}{a}\cos\left(\frac{n_x \pi x}{a}\right)\sin\left(\frac{n_y \pi y}{b}\right)\vetor{e}_y - \frac{n_y}{b}\sin\left(\frac{n_x \pi x}{a}\right)\cos\left(\frac{n_y \pi y}{b}\right)\vetor{e}_x\right] e^{i(kz - \omega t)}
    \end{equation*}
    são os campos no tubo.
\end{proof}
