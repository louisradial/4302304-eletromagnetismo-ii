% vim: spl=pt
\begin{exercício}{Partícula carregada em movimento retilíneo e uniforme}{ex3}
   Considere que uma partícula pontual de carga \(q\) se movimenta com velocidade constante \(\vetor{v} = v \vetor{e}_1\) segundo um referencial inercial \(\mathscr{K}.\) Por meio de uma transformação de Lorentz, encontre os potenciais \(\phi(\vetor{\x}, t)\) e \(\vetor{A}(\vetor{\x},t)\), assim como os campos \(\vetor{E}(\vetor{\x}, t)\) e \(\vetor{B}(\vetor{\x}, t)\).
\end{exercício}
\begin{proof}[Resolução]
   Vamos assumir que a partícula é massiva, isto é, que \(v \neq c.\) Em seu referencial de repouso \(\tilde{\mathscr{K}}\) em que a carga se encontra na origem, temos
   \begin{equation*}
      \tilde{\phi}(\vetor{\tilde{\x}}, \tilde{t}) = \frac{q}{4\pi \epsilon_0 \norm{\vetor{\tilde{\x}}}}
      \quad\text{e}\quad
      \vetor{\tilde{A}}(\vetor{\tilde{\x}}, \tilde{t}) = \vetor{0} \implies
      \vetor{\tilde{E}}(\vetor{\tilde{\x}},\tilde{t}) = \frac{q \vetor{\tilde{\x}}}{4\pi \epsilon_0 \norm{\vetor{\tilde{\x}}}^3} \quad\text{e}\quad
      \vetor{\tilde{B}}(\vetor{\tilde{\x}},\tilde{t}) = \vetor{0}.
   \end{equation*}
   Seja \(\Lambda\indices{^\mu_\nu}\) as componentes do boost que relaciona \(\mathscr{K}\) com \(\tilde{\mathscr{K}},\) isto é, \(x^\mu = \Lambda\indices{^\mu_\nu} \tilde{x}^\nu,\) temos
   \begin{align*}
      A^\mu = \Lambda\indices{^\mu_\nu} \tilde{A}^\nu &\implies \phi(\vetor{\x}, t) = \gamma \tilde{\phi}(\vetor{\tilde{\x}}, \tilde{t}) \quad\text{e}\quad \vetor{A}(\vetor{\x}, t) = -\gamma \beta \frac{\tilde{\phi}(\vetor{\tilde{\x}}, \tilde{t})}{c}\vetor{e}_1\\
                                                      &\implies \phi(\vetor{\x},t) = \frac{q}{4\pi \epsilon_0 \sqrt{(x + \beta ct)^2 + \frac{y^2 + z^2}{\gamma^2}}}
                                                      \quad\text{e}\quad \vetor{A}(\vetor{\x}, t)  = -\frac{\beta q}{4\pi \epsilon_0 c\sqrt{(x + \beta ct)^2 + \frac{y^2 + z^2}{\gamma^2}}}\vetor{e}_1
   \end{align*}
   como as expressões dos potenciais, onde \(\beta = \frac{v}{c}\) e \(\gamma = (1 - \beta^2)^{-\frac12}.\) As componentes do tensor eletromagnético \(F^{\mu\nu}\) são
   \begin{align*}
      F^{\mu\nu} &= \Lambda\indices{^\mu_\sigma}\Lambda^{\nu_\rho} \tilde{F}^{\sigma \rho}\\
                 &= \Lambda\indices{^\mu_0} \Lambda\indices{^\nu_r} \tilde{F}^{0r}  + \Lambda\indices{^\mu_s} \Lambda\indices{^\nu_0} \tilde{F}^{s0} + \Lambda\indices{^\mu_s}\Lambda\indices{^\nu_r} \tilde{F}^{sr}\\
                 &= \frac1c\left(\Lambda\indices{^\mu_0}\Lambda\indices{^\nu_k} - \Lambda\indices{^\mu_k}\Lambda\indices{^\nu_0}\right)\tilde{E}^{k} + \epsilon^{ksr} \Lambda\indices{^\mu_s} \Lambda\indices{^\nu_r} \tilde{B}_k\\
                 &= \left(\Lambda\indices{^\mu_0}\Lambda\indices{^\nu_k} - \Lambda\indices{^\mu_k}\Lambda\indices{^\nu_0}\right)\tilde{F}^{0k},
   \end{align*}
   portanto as componentes do campo elétrico são
   \begin{align*}
      E^1 &= cF^{01}&
      E^2 &= cF^{02}&
      E^3 &= cF^{03}\\
          &= c\left(\Lambda\indices{^0_0}\Lambda\indices{^1_k} - \Lambda\indices{^0_k}\Lambda\indices{^1_0}\right)\tilde{F}^{0k}&
          &= c\left(\Lambda\indices{^0_0}\Lambda\indices{^2_k} - \Lambda\indices{^0_k}\Lambda\indices{^2_0}\right)\tilde{F}^{0k}&
          &= c\left(\Lambda\indices{^0_0}\Lambda\indices{^3_k} - \Lambda\indices{^0_k}\Lambda\indices{^3_0}\right)\tilde{F}^{0k}\\
          &= c(\gamma \Lambda\indices{^1_k} + \gamma \beta \Lambda\indices{^0_k})\tilde{F}^{0k}&
          &= c\gamma \Lambda\indices{^2_k}\tilde{F}^{0k}&
          &= c\gamma \Lambda\indices{^2_k}\tilde{F}^{0k}\\
          &= c(\gamma^2 - \gamma^2 \beta^2) \tilde{F}^{01}&
          &= c\gamma \tilde{F}^{02}&
          &= c\gamma \tilde{F}^{03}\\
          &= \tilde{E}^1&
          &= \gamma \tilde{E}^2&
          &= \gamma \tilde{E}^3,
   \end{align*}
   e as do campo magnético são
   \begin{align*}
      B^1 &= F^{23}&
      B^2 &= F^{31}&
      B^3 &= F^{12}\\
          &= \left(\Lambda\indices{^2_0}\Lambda\indices{^3_k} - \Lambda\indices{^2_k}\Lambda\indices{^3_0}\right)\tilde{F}^{0k}&
          &= \left(\Lambda\indices{^3_0}\Lambda\indices{^1_k} - \Lambda\indices{^3_k}\Lambda\indices{^1_0}\right)\tilde{F}^{0k}&
          &= \left(\Lambda\indices{^1_0}\Lambda\indices{^2_k} - \Lambda\indices{^1_k}\Lambda\indices{^2_0}\right)\tilde{F}^{0k}\\
          &= 0&
          &= \gamma\beta \Lambda\indices{^3_k} \tilde{F}^{0k}&
          &= -\gamma\beta \Lambda\indices{^2_k} \tilde{F}^{0k}\\
          &&
          &= \gamma\beta \tilde{F}^{03}&
          &= - \gamma\beta \tilde{F}^{02}\\
          &&
          &= \frac{\gamma \beta E^3}{c}&
          &= - \frac{\gamma \beta E^2}{c}.
   \end{align*}
   Dessa forma, 
   \begin{equation*}
      \vetor{E}(\vetor{\x}, t) = \frac{\gamma q \left[(x + \beta ct)\vetor{e}_1 + y\vetor{e}_2 + z\vetor{e}_3\right]}{4\pi \epsilon_0 \left[\gamma^2 (x + \beta ct)^2 + y^2 + z^2\right]^{\frac32}}
      \quad\text{e}\quad
      \vetor{B}(\vetor{\x}, t) = \frac{\gamma \beta q \left(z\vetor{e}_2 - y\vetor{e}_3\right)}{4\pi \epsilon_0 c\left[\gamma^2 (x + \beta ct)^2 + y^2 + z^2\right]^{\frac32}}
   \end{equation*}
   são os campos elétrico e magnético no referencial \(\mathscr{K}.\)
\end{proof}
