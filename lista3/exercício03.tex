% vim: spl=pt
\begin{exercício}{Potenciais retardados satisfazem o gauge de Lorenz}{ex3}
   Mostre que os potenciais retardados, 
   \begin{equation*}
      \phi(\vetor{\x}, t) = \frac{1}{4\pi \epsilon_0}\int_{\mathbb{R}^3} \dln3{\x'} \frac{\rho(\vetor{\x'}, t_r)}{\norm{\D}}
      \quad\text{e}\quad
      \vetor{A}(\vetor{\x},t) = \frac{\mu_0}{4\pi} \int_{\mathbb{R}^3} \dln3{\x'} \frac{\vetor{J}(\vetor{\x'}, t_r)}{\norm{\D}},
   \end{equation*}
   sempre satisfazem a condição do gauge de Lorenz.
\end{exercício}
\begin{proof}[Resolução]
    
\end{proof}
