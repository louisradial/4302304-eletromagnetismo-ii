% vim: spl=pt
\begin{exercício}{Potenciais de Liénard-Wiechert para uma carga em movimento uniforme}{ex2}
   Consideremos uma partícula com carga \(q\) em movimento uniforme, \(\vetor{\x}_q(t) = t\vetor{v},\) com velocidade \(\vetor{v}\) constante.
   \begin{enumerate}[label=(\alph*)]
      \item Obtenha o tempo de retardo \(t_r\) associado a uma posição \(\vetor{\x}\) e instante \(t\) de observação. A partir disso, escreva explicitamente os potenciais de Liénard-Wiechert \(\phi(\vetor{\x}, t)\) e \(\vetor{A}(\vetor{\x}, t).\)
      \item Verifique que os potenciais encontrados no item anterior satisfazem o gauge de Lorenz.
      \item Considere que o movimento da carga é tal que \(\vetor{v} = v \vetor{e}_z.\) Encontre os potenciais e a componente \(x\) do campo elétrico no ponto \(b\vetor{e}_x\) no instante \(t\).
   \end{enumerate}
\end{exercício}
\begin{proof}[Resolução]
    
\end{proof}
