% vim: spl=pt
\begin{theorem}{Potenciais de Liénard-Wiechert}{lw}
   Os potenciais para uma partícula com carga \(q\) em movimento segundo a trajetória \(\vetor{\x}_q(t)\) são dados por
   \begin{equation*}
      \phi(\vetor{\x}, t) = \frac{q}{4\pi \epsilon_0}\left[\frac{1}{R - \vetor{\beta} \cdot \vetor{R}}\right]_{\ret}
      \quad\text{e}\quad
      \vetor{A}(\vetor{\x}, t) = \frac{q}{4\pi \epsilon_0 c}\left[\frac{\vetor{\beta}}{R - \vetor{\beta}\cdot \vetor{R}}\right]_{\ret},
   \end{equation*}
   onde
   \begin{equation*}
      \vetor{\beta}(t) = \frac1c \diff{\vetor{\x}_q}{t},\quad
      \vetor{R}(\vetor{\x},t) = \vetor{\x} - \vetor{\x}_q(t),\quad
      R(\vetor{\x}, t) = \norm{\vetor{R}(\vetor{\x}, t)},
   \end{equation*}
   e \([f(\vetor{\x}, t)]_\ret = f(\vetor{\x}, \tr(\vetor{\x}, t)),\) em que 
   \begin{equation*}
      \tr = t - \frac{\norm{\vetor{\x} - \vetor{\x}_q(\tr)}}{c}
   \end{equation*}
   é a equação que define unicamente o tempo de retardo \(\tr(\vetor{\x}, t).\)
\end{theorem}
\begin{proof}
   Notemos que 
   \begin{align*}
      \diff*{\left(t' - t + \frac{\norm{\vetor{\x} - \vetor{\x}_q(t')}}{c}\right)}{t'} 
      &= 1 - \frac{\diff{\vetor{\x}_q}{t'}\cdot\left[\vetor{\x} - \vetor{\x}_q(t')\right]}{c \norm{\vetor{\x} - \vetor{\x}_q(t')}}\\
      &= 1 - \frac{\vetor{\beta}(t') \cdot \vetor{R}(\vetor{\x}, t')}{R(\vetor{\x}, t')} > 0
   \end{align*}
   pois temos \(\norm{\vetor{\beta(t')}} < 1\) para todo \(t'.\) Assim, podemos inferir que a aplicação \(t' \mapsto t' - t + \frac{\norm{\vetor{\x} - \vetor{\x}_q(t')}}{c}\) é injetora, portanto a equação \(t' = t - \frac{\norm{\vetor{\x} - \vetor{\x}_q(t')}}{c}\) admite no máximo uma solução, que definimos como \(\tr(\vetor{\x}, t).\) Ainda, concluímos que
   \begin{equation*}
      \delta\left[t' - t + \frac{\norm{\vetor{\x} - \vetor{\x}_q(t')}}{c}\right] = \left[\frac{R}{R - \vetor{\beta} \cdot \vetor{R}}\right]_{\ret} \delta\left[t' - \tr(\vetor{\x}, t)\right],
   \end{equation*}
   pela identidade
   \begin{equation*}
      \delta(g(x)) = \sum_n \frac{1}{\abs{g'(x_n)}} \delta(x - x_n),
   \end{equation*}
   para \(g(x_n) = 0\) e \(g'(x_n) \neq 0.\)

   No gauge de Lorenz temos
   \begin{align*}
      \phi(\vetor{\x}, t) &= \frac{1}{4\pi \epsilon_0}\int_{\mathbb{R}^3} \dln3{\x'} \frac{q\delta\left[\vetor{\x'} - \vetor{\x}_q\left(t - \frac{\norm{\D}}{c}\right)\right]}{\norm{\D}}\\
                          &= \frac{q}{4\pi \epsilon_0}\int_{\mathbb{R}^3} \dln3{\x'} \int_{\mathbb{R}} \dli{t'} \frac{\delta\left[\vetor{\x'} - \vetor{\x}_q(t')\right]}{\norm{\D}} \delta\left(t' - t + \frac{\norm{\D}}{c}\right)\\
                          &= \frac{q}{4\pi \epsilon_0} \int_{\mathbb{R}} \dli{t'} \delta\left(t' - t + \frac{\norm{\D}}{c}\right) \int_{\mathbb{R}^3} \dln3{\x'} \frac{\delta\left[\vetor{\x'} - \vetor{\x}_q(t')\right]}{\norm{\D}}\\
                          &= \frac{q}{4\pi \epsilon_0} \int_{\mathbb{R}} \dli{t'} \frac{\delta\left(t' - t + \frac{\norm{\vetor{\x} - \vetor{\x}_q(t')}}{c}\right)}{\norm{\vetor{\x'} - \vetor{\x}_q(t')}}\\
                          &= \frac{q}{4\pi \epsilon_0} \left[\frac{R}{R - \vetor{\beta} \cdot \vetor{R}}\right]_\ret \int_{\mathbb{R}} \dli{t'} \frac{\delta(t' - \tr)}{\norm{\vetor{\x'} - \vetor{\x}_q(t')}}\\
                          &= \frac{q}{4\pi \epsilon_0} \left[\frac{1}{R - \vetor{\beta} \cdot \vetor{R}}\right]_\ret
   \end{align*}
   e
   \begin{align*}
      \vetor{A}(\vetor{\x}, t) &= \frac{\mu_0}{4\pi}\int_{\mathbb{R}^3} \dln3{\x'} \frac{qc\vetor{\beta}\left(t - \frac{\norm{\D}}{c}\right)\delta\left[\vetor{\x'} - \vetor{\x}_q\left(t - \frac{\norm{\D}}{c}\right)\right]}{\norm{\D}}\\
                               &= \frac{q}{4\pi \epsilon_0 c} \int_{\mathbb{R}} \dli{t'} \vetor{\beta}(t')\delta\left(t' - t + \frac{\norm{\D}}{c}\right) \int_{\mathbb{R}^3} \dln3{\x'} \frac{\delta\left[\vetor{\x'} - \vetor{\x}_q(t')\right]}{\norm{\D}}\\
                               &= \frac{q}{4\pi \epsilon_0 c} \int_{\mathbb{R}} \dli{t'} \frac{\vetor{\beta}(t')\delta\left(t' - t + \frac{\norm{\vetor{\x} - \vetor{\x}_q(t')}}{c}\right)}{\norm{\vetor{\x'} - \vetor{\x}_q(t')}}\\
                               &= \frac{q}{4\pi \epsilon_0 c} \left[\frac{R}{R - \vetor{\beta} \cdot \vetor{R}}\right]_\ret \int_{\mathbb{R}} \dli{t'} \frac{\vetor{\beta}(t')\delta(t' - \tr)}{\norm{\vetor{\x'} - \vetor{\x}_q(t')}}\\
                               &= \frac{q}{4\pi \epsilon_0c} \left[\frac{\vetor{\beta}}{R - \vetor{\beta} \cdot \vetor{R}}\right]_\ret,
\end{align*}
   como desejado.
\end{proof}

\begin{theorem}{Campos associados aos potenciais de Liénard-Wiechert}{lw2}
   Os campos associados ao movimento de uma carga \(q\) segundo a trajetória \(\vetor{\x}_q(t)\) são dados por
   \begin{equation*}
      \vetor{E}(\vetor{\x}, t) = \frac{q}{4\pi \epsilon_0}\left[\frac{\vetor{n} - \vetor{\beta}}{\gamma^2 (1 - \vetor{\beta} \cdot \vetor{n})^3 R^2} + \frac{\vetor{n} \times [(\vetor{n} - \vetor{\beta}) \times \dot{\vetor{\beta}}]}{(1 - \vetor{\beta}\cdot \vetor{n})^3 R}\right]_\ret
      \quad\text{e}\quad
      \vetor{B}(\vetor{\x},t) = \frac1c \vetor{n}(\vetor{\x}, \tr) \times \vetor{E}(\vetor{\x}, t),
   \end{equation*}
   onde \(\gamma(t) = \left[1 - \norm{\vetor{\beta}(t)}^2\right]^{-\frac12}\) e \(\vetor{n}(\vetor{\x}, t) = \frac{\vetor{\x} - \vetor{\x}_q(t)}{\norm{\vetor{\x} - \vetor{\x}_q(t)}}\).
\end{theorem}
\begin{proof}
   Derivando a equação implícita que define o tempo de retardo, obtemos
   \begin{align*}
      \diffp{\tr}{x^\mu} = \frac{1}{c}\delta^0_\mu - \frac{1}{c} \frac{\left(\delta^j_\mu\vetor{e}_j - c \vetor{\beta}(\tr) \diffp{\tr}{x^\mu}\right)\cdot \vetor{R}(\vetor{\x}, \tr)}{R(\vetor{\x}, \tr)} 
      % &\implies \diffp{\tr}{x^\mu} = \frac{\frac{1}{c} \delta^0_\mu - \frac1c \delta_{\mu}^j \inner{\vetor{e}_j}{\vetor{n}(\vetor{\x}, \tr)}}{1 - \vetor{\beta}(\tr) \cdot \vetor{n}(\vetor{\x}, \tr)}\\
      &\implies \diffp{\tr}{x^\mu} = \frac1c \left[\frac{\delta^0_\mu - \delta^j_\mu \inner{\vetor{e}_j}{\vetor{n}}}{1 - \vetor{\beta} \cdot \vetor{n}}\right]_\ret,
   \end{align*}
   onde \(\vetor{n}(\vetor{\x}, t) = \frac{\vetor{R}(\vetor{\x}, t)}{{R}(\vetor{\x}, t)}\), isto é,
   \begin{equation*}
      \diffp{\tr}{t} = \left[\frac{1}{1 - \vetor{\beta} \cdot \vetor{n}}\right]_\ret
      \quad\text{e}\quad
      \diffp{\tr}{x^i} = -\frac1c\inner*{\vetor{e}_i}{\left[\frac{\vetor{n}}{1 - \vetor{\beta} \cdot \vetor{n}}\right]_\ret}.
   \end{equation*}
   Como \(c(t - \tr) = R(\vetor{\x}, \tr),\) temos \(\partial_\mu R(\vetor{\x}, \tr) = \delta^0_\mu - c\partial_\mu\tr,\) ou seja,
   \begin{equation*}
      \diffp*{R(\vetor{\x}, \tr)}{t} = - \left[\frac{c\vetor{\beta}\cdot \vetor{n}}{1 - \vetor{\beta}\cdot \vetor{n}}\right]_\ret
      \quad\text{e}\quad
      \diffp*{R(\vetor{\x},\tr)}{x^i} = \inner*{\vetor{e}_i}{\left[\frac{\vetor{n}}{1 - \vetor{\beta} \cdot \vetor{n}}\right]_\ret}.
   \end{equation*}
   Com isso, temos
   \begin{align*}
      \diffp*{\left[R - \vetor{\beta} \cdot \vetor{R}\right]_\ret}{x^\mu} &= \delta^0_\mu - c \diffp{\tr}{x^\mu} - \left[\dot{\vetor{\beta}} \cdot \vetor{R} - c \norm{\vetor{\beta}}^2\right]_\ret \diffp{\tr}{x^\mu} - \delta_\mu^j\inner{\vetor{e}_j}{\vetor{\beta}(\tr)}\\
                                                                          &= \delta^0_\mu - \delta^j_\mu \inner{\vetor{e_j}}{\vetor{\beta}(\tr)} - \left[\dot{\vetor{\beta}} \cdot \vetor{R} + \frac{c}{\gamma^2}\right]_\ret \left\{\frac1c\left[\frac{\delta^0_\mu - \delta^j_\mu \inner{\vetor{e_j}}{\vetor{n}}}{1 - \vetor{\beta}\cdot\vetor{n}}\right]_{\ret}\right\}\\
                                                                          &= \delta_\mu^0\left[\frac{\norm{\vetor{\beta}}^2 - \vetor{\beta}\cdot \vetor{n} - \dot{\vetor{\beta}}\cdot \frac{\vetor{R}}{c}}{1 - \vetor{\beta}\cdot \vetor{n}}\right]_\ret + \delta^j_\mu \inner*{\vetor{e_j}}{\left[\frac{\left(\dot{\vetor{\beta}} \cdot \frac{\vetor{R}}{c} + \frac{1}{\gamma^2}\right)\vetor{n}-(1 - \vetor{\beta}\cdot \vetor{n})\vetor{\beta}}{1 - \vetor{\beta} \cdot \vetor{n}}\right]_\ret},
   \end{align*}
   onde definimos \(\gamma(t) = \left[1 - \norm{\vetor{\beta}(t)}^2\right]^{-\frac12}\).

   Com estes resultados, obtemos
   \begin{align*}
      \nabla \phi(\vetor{\x}, t) &= -\frac{q}{4\pi \epsilon_0}\frac{\vetor{e}_i}{\left[(1 - \vetor{\beta} \cdot \vetor{n})^2 R^2\right]_\ret} \diffp*{[R - \vetor{\beta} \cdot \vetor{R}]_\ret}{x^i}\\
                                 &=-\frac{q}{4\pi \epsilon_0}\frac{\vetor{e}_i}{\left[(1 - \vetor{\beta} \cdot \vetor{n})^2 R^2\right]_\ret} \inner*{\vetor{e_i}}{\left[\frac{\left(\dot{\vetor{\beta}} \cdot \frac{\vetor{R}}{c} + \frac{1}{\gamma^2}\right)\vetor{n}-(1 - \vetor{\beta}\cdot \vetor{n})\vetor{\beta}}{1 - \vetor{\beta} \cdot \vetor{n}}\right]_\ret}\\
                                 &=-\frac{q}{4\pi \epsilon_0}\left[\frac{\left(\frac1c \dot{\vetor{\beta}} \cdot \vetor{R} + \frac{1}{\gamma^2}\right)\vetor{n}-(1 - \vetor{\beta}\cdot \vetor{n})\vetor{\beta}}{(1 - \vetor{\beta} \cdot \vetor{n})^3 R^2}\right]_\ret,
   \end{align*}
   \begin{align*}
      \diffp{\vetor{A}(\vetor{\x},t)}{t} &= \frac{q }{4\pi \epsilon_0 c}\left[\frac{\dot{\vetor{\beta}}}{(1 - \vetor{\beta}\cdot \vetor{n})R}\right]_\ret \diffp{\tr}{t} - \frac{q}{4\pi \epsilon_0}\left[\frac{\vetor{\beta}}{(1 - \vetor{\beta} \cdot \vetor{n})^2R^2}\right]_\ret \diffp*{[R - \vetor{\beta}\cdot \vetor{R}]_\ret}{x^0}\\
                                         &= \frac{q}{4\pi \epsilon_0 c} \left[\frac{\dot{\vetor{\beta}}}{(1 - \vetor{\beta} \cdot \vetor{n})^2 R}\right]_\ret - \frac{q}{4\pi \epsilon_0} \left[\frac{\norm{\vetor{\beta}}^2 - \vetor{\beta}\cdot \vetor{n} - \frac1c\dot{\vetor{\beta}}\cdot \vetor{R}}{(1 - \vetor{\beta} \cdot \vetor{n})^3 R^2}\vetor{\beta}\right]_\ret\\
                                         &=-\frac{q}{4\pi \epsilon_0} \left[\frac{(\norm{\vetor{\beta}}^2 - \vetor{\beta} \cdot \vetor{n} - \frac1c\dot{\vetor{\beta}} \cdot \vetor{R}) \vetor{\beta} - \frac1c(1 - \vetor{\beta}\cdot \vetor{n}) R \dot{\vetor{\beta}}}{(1 - \vetor{\beta}\cdot \vetor{n})^3 R^2}\right]_\ret,
   \end{align*}
   logo o campo elétrico é dado por
   \begin{align*}
      \vetor{E}(\vetor{\x}, t) &= \frac{q}{4\pi \epsilon_0}\left[\frac{\frac1c \dot{\vetor{\beta}}\cdot \vetor{R} + \frac{1}{\gamma^2}}{(1 - \vetor{\beta}\cdot \vetor{n})^3 R^2}\vetor{n} 
      + \frac{\norm{\vetor{\beta}}^2 - 1 - \frac1c\dot{\vetor{\beta}} \cdot\vetor{R} }{(1 - \vetor{\beta}\cdot \vetor{n})^3 R^2}\vetor{\beta} 
   + \frac{\frac1c(\vetor{\beta} \cdot \vetor{n} - 1) R}{(1 - \vetor{\beta}\cdot \vetor{n})^3 R^2} \dot{\vetor{\beta}}\right]_\ret\\
                               &= \frac{q}{4\pi \epsilon_0} \left[\frac{\vetor{n} - \vetor{\beta}}{\gamma^2 (1 - \vetor{\beta} \cdot \vetor{n})^3 R^2} + \frac{(\dot{\vetor{\beta}}\cdot\vetor{n}) (\vetor{n} - \vetor{\beta}) - (\vetor{n}\cdot\vetor{n} - \vetor{\beta} \cdot \vetor{n}) \dot{\vetor{\beta}}}{c(1 - \vetor{\beta}\cdot \vetor{n})^3 R}\right]_\ret\\
                               &= \frac{q}{4\pi \epsilon_0} \left[\frac{\vetor{n} - \vetor{\beta}}{\gamma^2 (1 - \vetor{\beta} \cdot \vetor{n})^3 R^2} + \frac{\vetor{n} \times [(\vetor{n} - \vetor{\beta}) \times \dot{\vetor{\beta}}]}{c(1 - \vetor{\beta}\cdot \vetor{n})^3 R}\right]_\ret.
   \end{align*}
   Para o campo magnético, temos
   \begin{align*}
      \epsilon_{ijk}\diffp{A_k(\vetor{\x}, t)}{x^j} &= \frac{q \epsilon_{ijk}}{4\pi \epsilon_0 c}\left[\frac{\dot{\beta}_k}{(1 - \vetor{\beta}\cdot \vetor{n})R}\right]_\ret \diffp{\tr}{x^j} - \frac{q \epsilon_{ijk}}{4\pi \epsilon_0 c}\left[\frac{\beta_k}{(1 - \vetor{\beta} \cdot \vetor{n})^2R^2}\right]_\ret \diffp*{[R - \vetor{\beta}\cdot \vetor{R}]_\ret}{x^j}\\
                                                    &= -\frac{q \epsilon_{ijk}}{4\pi \epsilon_0 c^2}\left[\frac{ n_j \dot{\beta}_k}{(1 - \vetor{\beta}\cdot \vetor{n})^2 R} 
                                                    + \frac{\left(\dot{\vetor{\beta}}\cdot \vetor{R} + \frac{c}{\gamma^2}\right)n_j\beta_k - c(1 - \vetor{\beta} \cdot \vetor{n}) \beta_j \beta_k}{(1 - \vetor{\beta} \cdot \vetor{n})^3 R^2}\right]_\ret\\
                                                    &= - \frac{q \epsilon_{ijk}}{4\pi \epsilon_0 c^2}\left[\frac{cn_j \beta_k}{\gamma^2 (1 - \vetor{\beta} \cdot \vetor{n})^3 R^2} + \frac{(1 - \vetor{\beta} \cdot \vetor{n}) n_j \dot{\beta}_k + (\dot{\vetor{\beta}} \cdot \vetor{n}) n_j \beta_k}{(1 - \vetor{\beta} \cdot \vetor{n})^3 R}\right]_\ret\\
                                                    &= \frac{q \epsilon_{ijk} n_j}{4\pi \epsilon_0 c}\left[\frac{n_k - \beta_k}{\gamma^2 (1 - \vetor{\beta} \cdot \vetor{n})^3 R^2} - \frac{(1 - \vetor{\beta} \cdot \vetor{n}) \dot{\beta}_k + (\dot{\vetor{\beta}} \cdot \vetor{n}) \beta_k}{c(1 - \vetor{\beta} \cdot \vetor{n})^3 R}\right]_\ret,
   \end{align*}
   portanto como
   \begin{equation*}
      \vetor{n} \times \left\{\vetor{n} \times \left[(\vetor{n} - \vetor{\beta}) \times \dot{\vetor{\beta}}\right]\right\} = \vetor{n} \times \left[(\vetor{n} \cdot \dot{\vetor{\beta}})(\vetor{n} - \vetor{\beta}) - (1 - \vetor{\beta} \cdot \vetor{n}) \dot{\vetor{\beta}}\right] = - \left[(\vetor{n} \cdot \dot{\vetor{\beta}}) \vetor{n} \times \vetor{\beta} + (1 - \vetor{\beta} \cdot \vetor{n}) \vetor{n} \times \dot{\vetor{\beta}}\right],
   \end{equation*}
   segue que
   \begin{equation*}
      \vetor{B}(\vetor{\x}, t) = \frac{\vetor{n}(\vetor{\x}, \tr)}{c} \times \vetor{E}(\vetor{\x},t),
   \end{equation*}
   como desejado.
\end{proof}

\begin{exercício}{Potenciais de Liénard-Wiechert para uma carga em movimento uniforme}{ex2}
   Consideremos uma partícula com carga \(q\) em movimento uniforme, \(\vetor{\x}_q(t) = t\vetor{v},\) com velocidade \(\vetor{v}\) constante.
   \begin{enumerate}[label=(\alph*)]
      \item Obtenha o tempo de retardo \(\tr\) associado a uma posição \(\vetor{\x}\) e instante \(t\) de observação. A partir disso, escreva explicitamente os potenciais de Liénard-Wiechert \(\phi(\vetor{\x}, t)\) e \(\vetor{A}(\vetor{\x}, t).\)
      \item Verifique que os potenciais encontrados no item anterior satisfazem o gauge de Lorenz.
      \item Considere que o movimento da carga é tal que \(\vetor{v} = v \vetor{e}_z.\) Encontre os potenciais e a componente \(x\) do campo elétrico no ponto \(b\vetor{e}_x\) no instante \(t\).
   \end{enumerate}
\end{exercício}
\begin{proof}[Resolução]
   Escrevamos \(\vetor{\beta} = \frac1c \vetor{v}\) então
   \begin{align*}
      \tr = t - \frac{\norm{\vetor{\x} - \vetor{\x}_q(\tr)}}{c} &\implies \norm*{\frac{\vetor{\x}}{c} - \vetor{\beta}  \tr}^2 = (\tr - t)^2\\
                                                                % &\implies \frac{\norm{\vetor{\x}}^2}{c^2} - 2 \frac{\vetor{\x}}{c}\cdot\vetor{\beta} \tr+ \norm{\vetor{\beta}}^2 \tr^2 = \tr^2 - 2 t \tr + t^2\\
                                                                &\implies (1 - \norm{\vetor{\beta}}^2) \tr^2 - 2 \left(t - \frac{\vetor{\x}}{c} \cdot \vetor{\beta}\right) \tr + t^2 - \frac{\norm{\vetor{\x}}^2}{c^2} = 0\\
                                                                &\implies \tr = \frac{t - \frac{\vetor{\x}}{c} \cdot \vetor{\beta} \pm \sqrt{\left(t - \frac{\vetor{\x}}{c} \cdot \vetor{\beta}\right)^2 - \left(1 - \norm{\vetor{\beta}}^2\right)\left(t^2 - \frac{\norm{\vetor{\x}}^2}{c^2}\right)}}{1 - \norm{\vetor{\beta}}^2}.
   \end{align*}
   Para decidir o sinal que acompanha a raiz, notemos que no limite \(\norm{\vetor{\beta}} \ll 1\) temos
   \begin{equation*}
      \tr = t \pm \frac{\norm{\vetor{\x}}}{c},
   \end{equation*}
   portanto o sinal negativo corresponde ao tempo de retardo, isto é,
   \begin{align*}
      \tr(\vetor{\x}, t) &= \frac{t - \frac{\vetor{\x}}{c} \cdot \vetor{\beta} - \sqrt{\left(t - \frac{\vetor{\x}}{c} \cdot \vetor{\beta}\right)^2 - \left(1 - \norm{\vetor{\beta}}^2\right)\left(t^2 - \frac{\norm{\vetor{\x}}^2}{c^2}\right)}}{1 - \norm{\vetor{\beta}}^2}\\
                         &= \gamma^2 \left[t - \frac{\vetor{\x}}{c} \cdot \vetor{\beta} - \sqrt{\left(t - \frac{\vetor{\x}}{c} \cdot \vetor{\beta}\right)^2 - \frac{1}{\gamma^2}\left(t^2 - \frac{\norm{\vetor{\x}}^2}{c^2}\right)}\right].
   \end{align*}
   Pelo \cref{thm:lw},
   \begin{equation*}
      \phi(\vetor{\x}, t) = \frac{q}{4\pi \epsilon_0}\left[\frac{1}{R - \vetor{\beta} \cdot \vetor{R}}\right]_{\ret}
      \quad\text{e}\quad
      \vetor{A}(\vetor{\x}, t) = \frac{\mu_0 q\vetor{v}}{4\pi}\left[\frac{1}{R - \vetor{\beta}\cdot \vetor{R}}\right]_{\ret}
   \end{equation*}
   são os potenciais. Pelo \cref{ex:ex3}, as expressões utilizadas realmente satisfazem o gauge de Lorenz. Notemos que
   \begin{align*}
      [R - \vetor{\beta} \cdot \vetor{R}]_\ret &= c(t - \tr) - \vetor{\beta} \cdot \vetor{\x} + c \norm{\vetor{\beta}}^2 \tr\\
                                               &= c \left[\left(t - \frac{\vetor{\x}}{c} \cdot \vetor{\beta}\right) - \left(1 - \norm{\vetor{\beta}}^2\right)\tr\right]\\
                                               &= c \sqrt{\left(t - \frac{\vetor{\x}}{c} \cdot \vetor{\beta}\right)^2 - \frac{1}{\gamma^2}\left(t^2 - \frac{\norm{\vetor{\x}}^2}{c^2}\right)},
   \end{align*}
   portanto
   \begin{equation*}
      \phi(\vetor{\x}, t) = \frac{q}{4\pi \epsilon_0 c \sqrt{\left(t - \frac{\vetor{\x}}{c} \cdot \vetor{\beta}\right)^2 - \frac{1}{\gamma^2}\left(t^2 - \frac{\norm{\vetor{\x}}^2}{c^2}\right)}}
      \quad\text{e}\quad
      \vetor{A}(\vetor{\x},t) = \frac{\mu_0 q \vetor{v}}{4\pi c \sqrt{\left(t - \frac{\vetor{\x}}{c} \cdot \vetor{\beta}\right)^2 - \frac{1}{\gamma^2}\left(t^2 - \frac{\norm{\vetor{\x}}^2}{c^2}\right)}}
   \end{equation*}
   são as expressões dos potenciais. Como
   \begin{equation*}
      [\vetor{R} - R \vetor{\beta}]_\ret = \vetor{\x} - c\vetor{\beta} \tr - c\vetor{\beta}(t - \tr) =  \vetor{\x} - \vetor{v} t,
   \end{equation*}
   o campo elétrico é dado por
   \begin{align*}
      \vetor{E}(\vetor{\x}, t) &= \frac{q}{4\pi \epsilon_0} \left[\frac{\vetor{R} - \vetor{\beta} R}{\gamma^2 (R - \vetor{\beta} \cdot \vetor{R})^3}\right]_\ret\\
                               &= \frac{q}{4\pi \epsilon_0} 
                               \frac{(\vetor{\x} - \vetor{v} t)(1 - \beta^2)}{c^3 \left[\left(t - \frac{\vetor{\x}}{c} \cdot \vetor{\beta}\right)^2 - \frac{1}{\gamma^2}\left(t^2 - \frac{\norm{\vetor{\x}}^2}{c^2}\right)\right]^{\frac32}},
   \end{align*}
   pelo \cref{thm:lw2}.

   Para \(\vetor{v} = v\vetor{e}_z\), temos \(\vetor{\beta} = \beta \vetor{e}_z,\) com \(\beta = \frac{v}{c}\) e então
   \begin{equation*}
      \phi(b \vetor{e}_x, t) = \frac{q}{4\pi \epsilon_0 c \sqrt{\beta^2t^2 + \frac{b^2}{c^2}(1 - \beta^2)}}
      \quad\text{e}\quad
      \vetor{A}(b \vetor{e}_x, t) = \frac{\mu_0 q \vetor{v}}{4\pi \epsilon_0 c \sqrt{\beta^2t^2 + \frac{b^2}{c^2}(1 - \beta^2)}}
   \end{equation*}
   são os potenciais no ponto \(b\vetor{e}_x\) no instante \(t\). Por fim, 
   \begin{equation*}
      \inner{\vetor{e}_x}{\vetor{E}(b\vetor{e}_x, t)} = \frac{q}{4\pi \epsilon_0} \frac{(1 - \beta^2)b}{c^3 \left[\beta^2 t^2 + \frac{b^2}{c^2} (1 - \beta^2)\right]^{\frac32}}
   \end{equation*}
   é a componente \(x\) do campo elétrico neste ponto.
\end{proof}
