% vim: spl=pt
\begin{exercício}{Forças entre partículas em movimento}{ex7}
   Uma partícula de carga \(q_1\) está em repouso na origem. Uma segunda partícula de carga \(q_2\) move-se ao longo do eixo \(z\) com velocidade constante \(v,\) de forma que sua trajetória é dada por \(\vetor{\x}_2(t) = vt \vetor{e}_z.\)
   \begin{enumerate}[label=(\alph*)]
      \item Encontre a força \(\vetor{F}_{2(1)}\) sobre a partícula 2, devido à partícula 1, no instante \(t\).
      \item Encontre a força \(\vetor{F}_{1(2)}\) sobre a partícula 1, devido à partícula 2, no instante \(t\). A terceira Lei de Newton se aplica neste caso?
      \item Mostre que 
         \begin{equation*}
            \vetor{F}_{1(2)}(t) + \vetor{F}_{2(1)}(t) = -\diff{\vetor{p}_\mathrm{EM}}{t},
         \end{equation*}
         em que \(\vetor{p}_{\mathrm{EM}}(t)\) é o momento linear associado aos campos, dado por
         \begin{equation*}
            \vetor{p}_{\mathrm{EM}}(t) = \mu_0 \epsilon_0 \int_{\mathbb{R}^3} \dln3{\x} \vetor{S}(\vetor{\x}, t),
         \end{equation*}
         sendo \(\vetor{S}\) o vetor de Poynting dos campos totais.
   \end{enumerate}
\end{exercício}
\begin{proof}[Resolução]
    
\end{proof}
