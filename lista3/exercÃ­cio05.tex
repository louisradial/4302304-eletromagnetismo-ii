% vim: spl=pt
\begin{exercício}{Caso particular das equações de Jefimenko}{ex5}
   Suponha que em um sistema a densidade de corrente \(\vetor{J}\) seja constante no tempo. Portanto, pela equação de continuidade, \(\rho(\vetor{\x}, t) = \rho(\vetor{\x}, 0) + \dot\rho(\vetor{\x}, 0)t\). Mostre que, nesse caso,
   \begin{equation*}
      \vetor{E}(\vetor{\x}, t) = \frac1{4\pi \epsilon_0} \int_{\mathbb{R}^3} \dln3{\x'} \frac{\rho(\vetor{\x'}, t) (\D)}{\norm{\D}^3},
   \end{equation*}
   ou seja, a Lei de Coulomb se aplica com a densidade de carga calculada no tempo não retardado.
\end{exercício}
\begin{proof}[Resolução]
    
\end{proof}
