% vim: spl=pt
\begin{exercício}{Incidência oblíqua de ondas planas em uma abertura circular}{ex3}
    Escreve a integral de Herlmholtz-Kirchhoff no caso em que a onda incidente é uma onda plana, do tipo \(\psi(\vetor{\x}) = \psi_0 e^{i \vetor{k} \cdot \vetor{\x}},\) com \(\psi_0\) uma constante complexa associada à amplitude da onda. Ao invés da onda plana estar se propagando na direção \(z,\) trate o caso em que o vetor de onda \(\vetor{k}\) é tal que a incidência sobre a parede \(\Pi\) que contém as aberturas seja oblíqua e caracterizada por um certo ângulo de incidência \(\theta_I\). Resolva o problema de difração de Fraunhofer para uma abertura circular de raio \(a\), e estude a posição dos mínimos do padrão de interferência.
\end{exercício}
\begin{proof}[Resolução]
    
\end{proof}
