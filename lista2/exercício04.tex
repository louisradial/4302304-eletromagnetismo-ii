% vim: spl=pt
\begin{exercício}{Estimativa da resolução angular de telescópios}{ex4}
    Uma forma elegante de tratar um problema de difração por uma abertura é substituí-lo por um problema de reflexão por um espelho com o mesmo formato: a luz \emph{refletida} pelo espelho corresponde à luz que \emph{passaria} pela abertura no problema original. Reciprocamente, isso permite que tratemos o contexto da aquisição de sinais por um telescópio, por exemplo, como sendo de certa forma análogo à difração por uma abertura circular.

    Considere um telescópio cujo espelho tenha um diâmetro \(D\) (vamos assumi-lo plano). Duas estrelas são refletidas pelo espelho de forma que elas aparecem separadas por um ângulo muito pequeno. Considere portanto que a luz incidente no telescópio é dada aproximadamente por ondas planas que se propagam em direções ligeiramente diferentes, vistas da Terra.
    \begin{enumerate}[label=(\alph*)]
        \item O \emph{critério de Rayleigh}, largamente utilizado no contexto de dispositivos ópticos, estabelece que dois objetos observados se tornam indistinguíveis caso o máximo principal de um dos padrões recaia dentro do anel formado pelo primeiro mínimo do outro padrão. Assim, considerando um comprimento de onda específico \(\lambda,\) determine a separação angular mínima entre as duas estrelas para que elas possam ser \emph{resolvidas} pelo telescópio. Essa seria uma estimativa da resolução angular do equipamento.
        \item Usando sua resposta do item anterior, compute a resolução angular dos telescópios Hubble e GMT, cujos diâmetros (dos espelhos primários) são \SI{2.4}{\meter} e \SI{25.4}{\meter}, respectivamente. Faça os cálculos considerando um comprimento de onda na faixa ultravioleta - visível, por exemplo \(\lambda = \SI{400}{\nano\meter}\), e outro na faixa do infravermelho próximo, por exemplo \(\lambda = \SI{1000}{\nano\meter}.\) Pesquise valores apresentados nas especificações técnicas desses telescópios e compare com suas respostas.
    \end{enumerate}
\end{exercício}
\begin{proof}[Resolução]
    
\end{proof}
