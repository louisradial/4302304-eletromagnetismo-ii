% vim: spl=pt
\begin{exercício}{Conservação de energia}{ex5}
    Considere uma onda plana monocromática \(\psi_0 e^{ikz}\) incidindo normalmente sobre uma superfície onde há uma abertura \(\Sigma\) de formato arbitrário, cuja área é denotada por \(A_{\Sigma}.\) Considere, por simplicidade, que a origem do sistema de coordenadas coincide com o centro geométrico da abertura. Nas aproximações de Kirchhoff, a energia por unidade de comprimento ao longo da direção \(z\) que atravessa a abertura em \(z = 0\) é proporcional a \(\abs{\psi_0}^2 A_\Sigma,\) de forma que podemos escrever
    \begin{equation*}
        \diff{U}{z}(z = 0) = \abs{\psi_0}^2 A_{\Sigma}.
    \end{equation*}
    As frentes de onda na região \(z > 0\) não são, a rigor, planas. No entanto, caso as intensidades sejam significativas apenas nas vizinhanças do eixo \(z,\) podemos aproximar \(\vetor{n}'\cdot \frac{\D}{\norm{\D}} \sim 1\) para todos os pontos no cálculo de \(\psi(\vetor{\x})\) e tratar as frentes de onda nessa vizinhança como aproximadamente planas para fins práticos. Nesse caso, mostre que em um anteparo plano, paralelo à superfície e ocupando uma posição \(z = D > 0\), a energia por unidade de comprimento, dada por
    \begin{equation*}
        \diff{U}{z}(z = D) = \int_{\mathbb{R}}\dli{x}\int_{\mathbb{R}} \dli{y} \abs{\psi(x, y, D)}^2,
    \end{equation*}
    é igual à \(\difs{U}{z}(z = 0).\)
\end{exercício}
\begin{proof}[Resolução]
    
\end{proof}
