% vim: spl=pt
\begin{exercício}{Conservação de energia}{ex5}
    Considere uma onda plana monocromática \(\psi_0 e^{ikz}\) incidindo normalmente sobre uma superfície onde há uma abertura \(\Sigma\) de formato arbitrário, cuja área é denotada por \(A_{\Sigma}.\) Considere, por simplicidade, que a origem do sistema de coordenadas coincide com o centro geométrico da abertura. Nas aproximações de Kirchhoff, a energia por unidade de comprimento ao longo da direção \(z\) que atravessa a abertura em \(z = 0\) é proporcional a \(\abs{\psi_0}^2 A_\Sigma,\) de forma que podemos escrever
    \begin{equation*}
        \diff{U}{z}(z = 0) = \abs{\psi_0}^2 A_{\Sigma}.
    \end{equation*}
    As frentes de onda na região \(z > 0\) não são, a rigor, planas. No entanto, caso as intensidades sejam significativas apenas nas vizinhanças do eixo \(z,\) podemos aproximar \(\vetor{n}'\cdot \frac{\D}{\norm{\D}} \sim 1\) para todos os pontos no cálculo de \(\psi(\vetor{\x})\) e tratar as frentes de onda nessa vizinhança como aproximadamente planas para fins práticos. Nesse caso, mostre que em um anteparo plano, paralelo à superfície e ocupando uma posição \(z = D > 0\), a energia por unidade de comprimento, dada por
    \begin{equation*}
        \diff{U}{z}(z = D) = \int_{\mathbb{R}}\dli{x}\int_{\mathbb{R}} \dli{y} \abs{\psi(x, y, D)}^2,
    \end{equation*}
    é igual à \(\difs{U}{z}(z = 0).\)
\end{exercício}
\begin{proof}[Resolução]
    Com as aproximações consideradas, temos
    \begin{equation*}
        \psi(\vetor{\x}) = -\frac{ik}{2\pi} \int_{\Sigma} \dln2{\x'} \psi_0 \frac{e^{ik \norm{\D}}}{\norm{\D}},
    \end{equation*}
    portanto na superfície \(z = D,\) temos
    \begin{align*}
        \abs{\psi(x,y,D)}^2 &= \left(\frac{k}{2\pi}\right)^2 \int_{\Sigma} \dln2{\x'_1} \int_{\Sigma} \dln2{\x'_2} \abs{\psi_0}^2 \frac{e^{ik\norm{\vetor{\x} - \vetor{\x'_1}} - ik\norm{\vetor{\x} - \vetor{\x'_2}}}}{\norm{\vetor{\x} - \vetor{\x'_1}} \norm{\vetor{\x} - \vetor{\x'_2}}}\\
                            &\simeq \left(\frac{k}{2\pi D}\right)^2 \abs{\psi_0}^2 \int_{\Sigma} \dln2{\x'_1} \int_{\Sigma}\dln2{\x'_2} \exp\left[\frac{(x - x_1')^2 - (x - x_2')^2 + (y - y_1')^2 - (y - y_2')^2}{\frac{2D}{ik}}\right]\\
                            &\simeq \left(\frac{k}{2\pi D}\right)^2 \abs{\psi_0}^2 \int_{\Sigma} \dln2{\x'_1} \int_{\Sigma}\dln2{\x'_2} \exp\left[\frac{x (x_2' - x_1') + y(x_2' - x_1')}{\frac{D}{ik}}\right],
    \end{align*}
    onde descartamos os termos de ordem superior pois estamos assumindo que as intensidades são significativas apenas na vizinhanças do eixo \(z\). Com isso,
    \begin{align*}
        \diff{U}{z}(z = D) &= \left(\frac{k}{2\pi D}\right)^2\abs{\psi_0}^2 \int_{\mathbb{R}} \dli{x} \int_{\mathbb{R}}\dli{y}  \int_{\Sigma} \dln2{\x'_1} \int_{\Sigma}\dln2{\x'_2} \exp\left[\frac{x (x_2' - x_1') + y(x_2' - x_1')}{\frac{D}{ik}}\right]\\
                           &= \abs{\psi_0}^2 \int_{\Sigma} \dln2{\x'_1} \int_{\Sigma}\dln2{\x'_2}\int_{\mathbb{R}} \dli{x}\frac{k}{2\pi D} \exp\left[\frac{ik x(x_2' - x_1')}{D}\right] \int_{\mathbb{R}} \dli{y}\frac{k}{2\pi D} \exp\left[\frac{ik y(y_2' - y_1')}{D}\right]\\
                           &=  \abs{\psi_0}^2 \int_{\Sigma} \dln2{\x_1'} \int_{\Sigma} \dln2{\x_2'} \delta(x'_2 - x'_1) \delta(y_2' - y_1')\\
                           &= \abs{\psi_0}^2 \int_{\Sigma} \dln2{\x_1'} \int_{\Sigma} \dln2{\x_2'} \delta(\vetor{\x_1'} - \vetor{\x_2'})\\
                           &= \abs{\psi_0}^2 \int_{\Sigma} \dln2{\x'}\\
                           &= \diff{U}{z} (z = 0),
    \end{align*}
    como desejado.
\end{proof}
