% vim: spl=pt
\begin{exercício}{Estimativa dos primeiros máximos de intensidade para a difração de Fresnel}{ex7}
    Utilize alguma ferramenta computacional de sua preferência e gere um gráfico da Espiral de Cornu. Baseado no seu gráfico estime os três primeiros máximos de intensidade da curva \(I(y)\) para \(y > 0.\)
\end{exercício}
\begin{proof}[Resolução]
    Podemos reescrever a intensidade como
    \begin{equation*}
        I(x,y) = \abs{\psi_0}^2 \frac{D^4}{4(x^2 + y^2 + D^2)^2} \abs*{\frac12 + \frac{e^{-\frac14 i\pi}}{\sqrt{2}} F\left(\sqrt{\frac{k}{\pi D}} y\right)}^2,
    \end{equation*}
    então estimaremos os primeiros máximos e mínimos de intensidade ao longo do eixo \(y\) como os pontos nos quais \(C\left(\sqrt{\frac{k}{\pi D}} y\right) = S\left(\sqrt{\frac{k}{\pi D}} y\right)\) para \(y > 0\).
    \begin{figure}[h]
        \centering
        \includegraphics[scale=0.8]{exercício07.png}
        \caption{Espiral de Cornu}
    \end{figure}

    Utilizando a parametrização \(w = F(t)\) para a espiral de Cornu, vemos que os primeiros máximos ocorrem em \(t = 1.2653,\) \(t = 2.3527,\) e \(t = 3.0856.\) Assim, 
    \begin{equation*}
        y_1 = 1.2653\sqrt{\frac{\pi D}{k}},\quad
        y_2 = 2.3527\sqrt{\frac{\pi D}{k}},\quad\text{e}\quad
        y_3 = 3.0856\sqrt{\frac{\pi D}{k}}
    \end{equation*}
    são as estimativas para os primeiros máximos de intensidade.
\end{proof}
