% vim: spl=pt
\begin{exercício}{Difração de Fresnel para abertura semi-plana}{ex6}
    Uma onda plana monocromática \(\psi_0 e^{ikz}\) incide normalmente sobre uma superfície opaca que ocupa todo o semi-plano \((x', y' < 0),\) como na figura abaixo. A luz é projetada numa tela a uma distância \(D\) da superfície. A intensidade da luz na tela obedece um padrão de difração \(I(y),\) sendo \(y\) a direção perpendicular à \(z\) paralela ao plano da figura.
    \begin{center}
        \begin{tikzpicture}[scale=0.6]

% Colors
\definecolor{lightblue}{RGB}{220,240,255}
\definecolor{Apertureblue}{RGB}{100,170,230}

% Parameters
\def\zAperture{0}
\def\zScreen{5}
\def\ApertureHeight{6}
\def\n{8}
\def\barWidth{0.1}
\def\spacing{0.3}
\def\height{6}

% Blue shaded region on the right
\fill[lightblue!60] ({\zAperture - (\n + 1.2)*(\barWidth + \spacing)},0) rectangle (\zScreen, \ApertureHeight);

% Background light cone
\fill[Sky!5] (\zAperture,0) -- (\zAperture, \height/2) -- (\zScreen, 0) -- cycle;

% Frentes de onda
\foreach \i in {0,...,\n} {
  \pgfmathsetmacro\x{\zAperture - (\i + 1)*(\barWidth+\spacing)}
  \fill[Apertureblue] (\x,0) rectangle ++(\barWidth,\ApertureHeight);
}

% Screen
\draw[thick] (\zScreen,0) -- (\zScreen,\ApertureHeight);

% 
\draw[dashed, ->] (\zAperture, \height/2) -- (1.1*\zScreen, \height/2) node[above] {\(z\)};

% Aperture edge
\draw[very thick] (\zAperture,0) -- (\zAperture,\ApertureHeight/2);

% Distance D
\draw[<->] (\zAperture, -0.5) -- (\zScreen, -0.5) node[above,midway] {\(D\)};
% \node[above] at ({0.5*(\zAperture+\zScreen)}, -0.5) {$D$};
\end{tikzpicture}
    \end{center}
    Obtenha a função \(I(y)\) em termos das integrais de Fresnel,
    \begin{equation*}
        F(\xi) = \int_0^\xi \dli{\zeta} \exp\left(i\frac{\pi}{2} \zeta^2\right) = C(\xi) + i S(\xi).
    \end{equation*}
    Sabendo que \(F(\xi) = - F(-\xi)\), que \(F(0) = 0,\) e que \(F(\xi) \to \frac{1 + i}{2}\) conforme \(\xi \to \infty\), estude o comportamento de \(I(y)\) para \(y \to \pm \infty\) e \(y = 0\). Expresse suas respostas em termos de \(I_0 = \abs{\psi_0}^2.\)
\end{exercício}
\begin{proof}[Resolução]
    Seja \(\Pi\) o plano \(z = 0\) que contém a superfície opaca, e seja \(\Sigma = \setc{(x', y') \in \Pi}{y' > 0}\) a abertura, então
    \begin{align*}
        \psi(\vetor{\x}) &= - \frac{i k}{2\pi} \int_{\Sigma} \dln2{\x'} \inner*{\vetor{n}'}{\frac{\D}{\norm{\D}}} \psi(\vetor{\x'}) \frac{e^{ik \norm{\D}}}{\norm{\D}}\\
                         &\simeq -\frac{ik \cos\theta_{\x}}{2\pi \norm{\vetor{\x}}} \int_{\Sigma} \dln2{\x'} \psi(\vetor{\x'}) e^{ik \norm{\D}}\\
                         &\simeq -\frac{ik \psi_0 \cos\theta_{\x}}{2\pi \norm{\vetor{\x}}} \int_{\Sigma} \dln2{\x'} e^{ik \norm{\D}}.
    \end{align*}
    Para \(\vetor{\x'} = x' \vetor{e}_x + y' \vetor{e}_y \in \Sigma\) e \(\vetor{\x} = x \vetor{e}_x + y \vetor{e}_y + z \vetor{e}_z\) na região de Fresnel, aproximamos
    \begin{equation*}
        \norm{\D} \simeq z + \frac{(x - x')^2 + (y - y')^2}{2z}.
    \end{equation*}
    Assim,
    \begin{align*}
        \psi(\vetor{\x}) &= -\frac{ik \psi_0 \cos\theta_{\x}}{2\pi \norm{\vetor{\x}}} e^{ikz} \int_{\mathbb{R}} \dli{x'} \int_{0}^\infty \dli{y'}e^{ik \frac{(x - x')^2 + (y - y')^2}{2z}}\\
                         &= -\frac{iz \psi_0 e^{ikz} \cos\theta_{\x}}{2\norm{\vetor{\x}}}  \int_{\mathbb{R}} \dli{\xi} \exp\left(i\frac{\pi}{2}\xi^2\right) \int_{-\sqrt{\frac{k}{\pi z}}y}^{\infty} \dli{\eta}\exp\left(i \frac{\pi}{2} \eta^2\right)\\
                         &= -\frac{i (1 + i)z^2 \psi_0 e^{ikz}}{2 \norm{\vetor{\x}}^2}  \left[\frac{1 + i}{2} + \int_{-\sqrt{\frac{k}{\pi z}}y}^0 \dli\eta \exp\left(i\frac{\pi}{2} \eta^2\right)\right]\\
                         &= \frac{z^2 \psi_0 e^{ikz}}{\norm{\vetor{\x}}^2} \left[\frac{1}{2} + \frac{1-i}{2}F\left(\sqrt{\frac{k}{\pi{z}}}y\right)\right]\\
                         % &= \psi_0 e^{ikz} \frac{z^2}{2 (x^2 + y^2 + z^2)}\left[\frac12 + \frac{1 - i}{2} F\left(\sqrt{\frac{k}{\pi z}} y\right)\right]\\
                         % &= \Psi_0\left[\frac12 + \frac{1 - i}{2} F\left(\sqrt{\frac{k}{\pi z}} y\right)\right]
    \end{align*}
    e então a intensidade na tela em \(z = D\) é dada por
    \begin{align*}
        I(x,y) &= \abs{\psi(x\vetor{e}_x + y\vetor{e}_y + D \vetor{e}_z)}^2\\
               &= \abs{\psi_0}^2 \frac{D^4}{(x^2 + y^2 + D^2)^2} \abs*{\frac12 + \frac{1 - i}{2} F\left(\sqrt{\frac{k}{\pi D}} y\right)}^2\\
               &= I_0 \frac{D^4}{(x^2 + y^2 + D^2)^2} \left\{\frac14 + \frac12\abs*{F\left(\sqrt{\frac{k}{\pi z}} y\right)}^2 + \Re\left[\frac{1 - i}{2} F\left(\sqrt{\frac{k}{\pi z}} y\right)\right]\right\}\\
               &= I_0 \frac{D^4}{(x^2 + y^2 + D^2)^2} \left[\frac14 + \frac12\abs*{F\left(\sqrt{\frac{k}{\pi z}} y\right)}^2 + \frac12 C\left(\sqrt{\frac{k}{\pi z}} y\right) + \frac12 S\left(\sqrt{\frac{k}{\pi z}} y\right)\right].
    \end{align*}
    A seguir, consideramos limites em que \(y \to \pm \infty\) e em que \(y \to 0\), mantendo constante o prefator \(c = \frac{D^4}{(x^2 + y^2 + D^2)^2}\), então 
    \begin{equation*}
        \lim_{\substack{y \to 0\\c\text{ constante}}}{I(x,y)} = c I_0 \abs*{\frac12 + \frac{1 - i}{2} F(0)}^2 = \frac14 cI_0
    \end{equation*}
    e
    \begin{equation*}
        \lim_{\substack{y \to \pm \infty\\c \text{ constante}}} I(x,y) = c I_0 \abs*{\frac12 + \frac{1 - i}{2} \left(\pm \frac{1 + i}{2}\right)}^2 = c I_0 \abs*{\frac12 \pm \frac12}^2 = \frac{(1 \pm 1)^2 c}{4}I_0,
    \end{equation*}
    isto é, \(I(y) \to 0\) quando \(y \to -\infty\) e \(I(y) \to c I_0\) quando \(y \to \infty.\)
\end{proof}
