% vim: spl=pt
\begin{exercício}{Rede de difração}{ex2}
    Uma onda emanada de uma fonte pontual incide sobre uma superfície contendo três aberturas retangulares iguais, de lados \(A\) e \(B\), contidas no plano \(z = 0\). A separação entre duas aberturas vizinhas é \(d\). Obtenha o padrão de difração no limite de Fraunhofer sobre um anteparo localizado em \(z = D,\) assumindo que a fonte está posicionada no eixo \(z,\) alinhada com o centro da abertura do meio. Em seguida, estude o caso de \(N\) aberturas iguais, ao invés de três. Assuma que \(N\) é um número ímpar e que a disposição das aberturas respeita a mesma simetria em relação ao eixo \(z\). Avalie as intensidades no anteparo em função de \(x\) e \(y\) e utilize uma ferramenta computacional de sua preferência para gerar figuras desses padrões.
\end{exercício}
\begin{proof}[Resolução]
    
\end{proof}
